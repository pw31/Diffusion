\documentclass[11pt]{article}
\usepackage{graphicx,natbib,amsmath}

\setlength{\textheight}{24.5cm}
\setlength{\textwidth}{16.5cm}
\setlength{\topmargin}{-14mm}
\setlength{\evensidemargin}{-2mm}
\setlength{\oddsidemargin}{-2mm}
\setlength{\headsep}{4mm}

\def\nH{n_{\langle\rm H\rangle}}
\def\Vl{V_{\ell}}
\def\Nl{N_{\ell}}
\def\rhod{\rho_{\rm d}}
\def\ek{\epsilon_k}
\def\pdiff#1#2{\frac{\partial #1}{\partial #2}}
\def\rhogr{\rho_{\rm gr}}
\def\Dd{D_{\!d}}

\def\apj{ApJ}
\def\pasp{PASP}
\def\pasj{PASJ}
\def\araa{ARA\&A}
\def\aap{A\&A}
\def\aaps{A\&AS}
\def\apjl{ApJL}
\def\apjs{ApJS}
\def\mnras{MNRAS}
\def\aj{AJ}
\def\nat{Nature}
\def\icarus{Icarus}
\def\pasa{PASA}

\def\la{\mathrel{\mathchoice {\vcenter{\offinterlineskip\halign{\hfil
$\displaystyle##$\hfil\cr<\cr\sim\cr}}}
{\vcenter{\offinterlineskip\halign{\hfil$\textstyle##$\hfil\cr
<\cr\sim\cr}}}
{\vcenter{\offinterlineskip\halign{\hfil$\scriptstyle##$\hfil\cr
<\cr\sim\cr}}}
{\vcenter{\offinterlineskip\halign{\hfil$\scriptscriptstyle##$\hfil\cr
<\cr\sim\cr}}}}}
\def\ga{\mathrel{\mathchoice {\vcenter{\offinterlineskip\halign{\hfil
$\displaystyle##$\hfil\cr>\cr\sim\cr}}}
{\vcenter{\offinterlineskip\halign{\hfil$\textstyle##$\hfil\cr
>\cr\sim\cr}}}
{\vcenter{\offinterlineskip\halign{\hfil$\scriptstyle##$\hfil\cr
>\cr\sim\cr}}}
{\vcenter{\offinterlineskip\halign{\hfil$\scriptscriptstyle##$\hfil\cr
>\cr\sim\cr}}}}}


\begin{document}

\centerline{Definitions}

\begin{tabular}{l|l|c}
\hline
symbol  & description & unit\\
\hline
$z$                       & vertical coordinate    & cm\\
$\ek$                     & abundance of element $k$ 
                            with respect to H      & --\\
$n_i$                     & particle density of molecule $i$ & cm$^{-3}$\\
$\nH$                     & total hydrogen nuclei density & cm$^{-3}$\\
$\rho = \sum_i n_i m_i = \mu_{\rm H}\,\nH = \mu \sum_i n_i$  
                          & gas mass density       & g\,cm$^{-3}$\\
$m_i$                     & mass of particle $i$   & g\\
$m_k$                     & mass of element $k$    & g\\
$\mu = {\sum_i n_i m_i}\Big/{\sum_i n_i}$ 
                          & mean gas particle mass & g\\
$\mu_{\rm H} = \sum_k \epsilon_k m_k$   
                          & proportionality constant & g\\
$T$                       & gas temperature        & K\\
$p = \frac{\rho}{\mu}kT = \sum_i n_i\,kT$  
                          & gas pressure           & dyn\,cm$^{-2}$\\
$D$                       & diffusion coefficient  & cm$^2$\,s$^{-1}$\\
\hline  
\end{tabular}

\bigskip
\section{The Atmospheric Diffusion Problem}

To simultaneously model the evolution of the chemical composition of
the atmosphere and the crust of a hot rocky planet, we use the
implicit/explicit time-dependent second-order diffusion solver {\sc
  Diffuse} developed by Peter Woitke.  At the top of the atmosphere, a
modified fixed-flux boundary condition will be applied to allow for
the Jeans escape of H$_2$ and He (see Sect.~\ref{UpperBound}), whereas
at the bottom, a modified fixed-concentration boundary condition will
be applied to treat the outgasing/deposition of elements to/from the
crust (see Sect.~\ref{LowerBound}). By book-keeping the element fluxed
through the lower boundary, we can also predict how the crust
composition slowly changes.
 
Between the boundaries, we solve the second-order diffusion problem
\begin{equation}
  \pdiff{(\nH\,\ek)}{t} + \nabla(\vec{v}\,\nH\ek) 
   ~=~ \nabla\left(\nH D\,\nabla\ek\right) \ .
\end{equation}
Assuming a 1-d plane-parallel and static $(\vec{v}=0)$ atmosphere, the
equations to solve are
\begin{equation}
  \frac{d}{dt}\Big(\nH\,\ek\Big) 
  ~=~ \frac{d}{dz}\left(\nH D\,\frac{d\ek}{dz}\right) \ .
  \label{eq:diff1}
\end{equation}
Assuming, in addition, a constant density structure
$\nH=\nH\!(z)$ the equations simplify to
\begin{equation}
  \frac{d\ek}{dt} 
  ~=~ \frac{1}{\nH}\,\frac{d}{dz}\!\left(\nH D\,\frac{d\ek}{dz}\right) \ ,
  \label{eq:diff2}
\end{equation}
where $\ek(z,t)$ are the height-dependent and time-dependent element
abundances we wish to determine. We note that $\nH=\nH\!(z)$ and
$D=D(z)$ vary by many orders of magnitude throughout the atmosphere,
often more than the element abundances we wish to determine, therefore 
we {\sl cannot} neglect the $\frac{d\nH}{dz}$ and $\frac{dD}{dz}$
terms as
\begin{equation}
  \frac{d\ek}{dt} ~\neq~ D\,\frac{d^{\,2}\ek}{dz^2} \ .
\end{equation}
The diffusion constant $D(z)$ is determined by proper microscopic diffusion
in the uppermost layers $D_{\rm micro}$, but otherwise mainly by
turbulent quasi-diffusion $D_{\rm mix}$ due to vertical mixing
processes excited by convective instabilities. 
\begin{equation}
  D(z) = D_{\rm micro}(z) + D_{\rm mix}(z)
\end{equation}
The turbulent diffusion takes place on large spatial scales, hence we
can safely assume that $D_{\rm mix}$ does not depend on the molecule
(or element) we want to diffuse. Concerning $D_{\rm micro}$, however, 
there is a small dependence on the size and reduced mass of the
molecule with respect to collisions with H$_2$, see
\citet[][Eq.~(26) therein]{Woitke2003}, which causes deviations
of $D$ as function of molecule by a factor of about 2 to 3. We will simply
neglect these deviations in the following.

This neglection allows us to consider the diffusion of elements
instead of calculating the diffusion of all individual molecules separately.
The total particle density $\rm[cm^{-3}]$ of element $k$ is given by 
\begin{equation}
  \nH\,\ek = \sum_i s_{i,k}\,n_i
\end{equation} 
where $s_{i,k}$ is the stoichiometic factor of element $k$ in molecule
$i$, for example $s_{\rm H_2O,H}=2$. Using Eq.~(\ref{eq:diff1}), the
diffusion of that total nuclei particle density is given by
\begin{eqnarray}
  \frac{d}{dt}\left(\nH\,\ek\right)
  &=& \frac{d}{dt}\sum_i s_{i,k}\,n_i
  ~=~ \sum_i s_{i,k}\,\frac{dn_i}{dt} \\
  &=& \sum_i s_{i,k}\,\frac{d}{dz}\!\left(\nH D\,\frac{d}{dz}
                     \left(\frac{n_i}{\nH}\right)\right) \\
  &=& \frac{d}{dz}\!\left(\nH D\,\frac{d}{dz}\left(\sum_i s_{i,k}
  \frac{n_i}{\nH}\right)\right) \\
  &=& \frac{d}{dz}\!\left(\nH D\,\frac{d\ek}{dz}\right)
\end{eqnarray}




\section{Spatial Grid and Initial Conditions}

We use an equidistant grid in $\{z_i\,|\,i\!=\!0,...,I\}$ to
discretise the 1d-structure of the planetary atmosphere with fixed
temperature and pressure points in hydrostatic equilibrium, $T(z)$
and $p(z)$.  At the lower boundary $(z=0)$ of the model, the
atmospheric gas is in contact with the crust. The crust is represented
by column densities $N_j^{\rm cond}$ for a number of $j=1,...,J$
condensed species.

At initialisation $(t=0)$, we consider one set of total (gas and
condensed) element abundances $\{\epsilon_k^0\}$ at temperature
$T(z=0)$ and pressure $p(z=0)$ in phase equilibrium, using the {\sc
  GGchem}-code \citep{Woitke2017}.  The total element abundances
$\{\epsilon_k^0\}$ are chosen from one of the sets listed in
Table~\ref{tab:eps}. The results of this phase equilibrium 
computation are
\begin{itemize}
\itemsep=-1pt
\parsep=0pt
  \item the number of simultaneously present condensates $J$,
  \item the selection of condensates,
  \item the volume density of units of the condensated species 
        $n_j^{\rm cond}\ (j=1,...,J)\ \rm[cm^{-3}]$,
  \item the element abundances $\{\epsilon_k^{\rm gas}\}$ of the gas
        above the condensates
\end{itemize}
We define the condensed element abundances as
\begin{equation}
  \nH\,\epsilon_k^{\rm cond} = \sum_{j=1}^J s_{j,k}\,n_j^{\rm cond}
\end{equation}
where $s_{j,k}$ is the stoichiometric factor of element $k$ in
condensate $j$, for example $s_{\rm Al_2O_3,O}=3$. The element
conservation then reads as
\begin{equation}
  \epsilon_k^0 = \epsilon_k^{\rm gas} + \epsilon_k^{\rm cond} \ .
\end{equation}
At initialisation, we fill the lowest atmospheric cell with element
composition $\epsilon_k^{\rm gas}$, and use the selection and ratio of
condensed species to initialise the crust. There is one free constant
in the phase equilibrium computations, namely the total amount of
condensates. If we add an arbitrary amount of our mixture of
condensates as
\begin{equation}
  \widetilde{\epsilon}_k^{\;0} = \epsilon_k^{\rm gas} 
                           + X\,\epsilon_k^{\rm cond} \ .
\end{equation}
and re-run the phase equilibrium computation for the same pressure
and temperature, but now with $\{\widetilde\epsilon_k^{\;0}\}$ instead of
$\{\epsilon_k^0\}$, all resulting gas properties including $\epsilon_k^{\rm
  gas}$ stay the same.  This property of phase equilibrium allows us
to consider the ``thickness of the active crust'' $D$ as a free 
parameter. The initial column densities in the active crust are
given by
\begin{equation}
  N_j^{\rm cond} = D\,n_j^{\rm cond}
\end{equation}
All atmospheric cells above the bottom cell $(i=2,...,I)$ are finally
filled with gas of solar abundances, or any other chosen atmospheric gas
element abundances.

\begin{table}
\centering
\caption{Different sets of element abundances normalised to hydrogen. 
$A(-B)$ means $A\times 10^{-B}$.}
\label{tab:eps}
\vspace*{2mm}
\begin{tabular}{cccc}
\hline
     &     solar & Earth crust & meteoritic \\
\hline
 H   & $1.00(+0)$ &   $1.00(+0)$ &   $1.00(+0)$ \\
 He  & $8.51(-2)$ &   $1.08(-6)$ &   -- \\
 Li  & $1.12(-11)$ &  $2.07(-3)$ &   $1.03(-5)$ \\
 C   & $2.69(-4)$ &   $1.20(-2)$ &   $5.25(-2)$ \\
 N   & $6.76(-5)$ &   $9.76(-4)$ &   $4.20(-3)$ \\
 O   & $4.90(-4)$ &   $2.07(+1)$ &   $1.05(+0)$ \\
 Na  & $1.74(-6)$ &   $7.39(-1)$ &   $1.00(-2)$ \\
 Mg  & $3.98(-5)$ &   $6.90(-1)$ &   $2.07(-1)$ \\
 Al  & $2.82(-6)$ &   $2.20(+0)$ &   $1.42(-2)$ \\
 Si  & $3.24(-5)$ &   $7.23(+0)$ &   $2.09(-1)$ \\
 P   & $2.57(-7)$ &   $2.44(-2)$ &   $1.49(-3)$ \\
 S   & $1.32(-5)$ &   $7.86(-3)$ &   $5.24(-2)$ \\
 Cl  & $3.16(-7)$ &   $2.94(-3)$ &   $4.38(-4)$ \\
 K   & $1.07(-7)$ &   $3.85(-1)$ &   $7.52(-4)$ \\
 Ca  & $2.19(-6)$ &   $7.45(-1)$ &   $1.15(-2)$ \\
 Ti  & $8.91(-8)$ &   $8.42(-2)$ &   $4.74(-4)$ \\
 Cr  & $4.37(-7)$ &   $1.41(-3)$ &   $2.42(-3)$ \\
 Mn  & $2.69(-7)$ &   $1.24(-2)$ &   $2.06(-3)$ \\
 Fe  & $3.16(-5)$ &   $7.26(-1)$ &   $1.65(-1)$ \\
 Ni  & $1.66(-6)$ &   $1.03(-3)$ &   $9.30(-3)$ \\
\hline
\end{tabular}
\end{table}


\section{Boundary Conditions}
\subsection{Upper boundary condition}
\label{UpperBound}

We specify the escaping element fluxes from the top of the atmosphere 
by applying the formula for Jeans escape $\to$ \citet{Tian2015} as
upper boundary condition of the model.

\noindent Where to apply that formula? $\to$ exobase $\to$ \citet{Volkov2011}.

\noindent Here, we need to decompose the elements into molecules,
apply the Jeans escape to each molecule, and lump the escaping fluxes
together to get the escaping element fluxes. 

\subsection{Lower boundary condition}
\label{LowerBound}

At the bottom of the atmosphere, the atmospheric gas is in contact
with the crust. At pressures between fractions of a bar to
several bars, the collision rates of gas particles with the crust are
huge, leading to a fast relaxation towards phase equilibrium.  We will
therefore assume that the gas at the bottom of the atmosphere is
saturated (is in phase equilibrium) with respect to the solid/liquid 
materials present in the crust.
\begin{equation}
  S_j(\ek)\Big|_{z=0} ~=~ 1
  \label{phaseEq}
\end{equation}
As shown in Appendix B of \citep{Woitke2017}, the number $J$ of
simultaneously present condensates in phase equilibrium is limited
and cannot exceed the number of elements $K$ contained in them,\\
\begin{center}
\begin{tabular}{rcl}
  $J$ &$=$& number of condensates in crust $j=1,...,J$\\
  $K$ &$=$& number of condensed elements in crust $k=1,...,K$ \\
  $J$ &$\leq$ & $K$ \\
  $N$ &$=$& $K-J~\geq~0$ ~number of non-limiting elements,
\end{tabular}
\end{center} 
i.e.\ the number of condensing elements can (and usually will) exceed
the number of condensates in the crust. In oder to formulate the inner
boundary condition for the diffusion experiment, we need to solve $J$
equations (Eq.~\ref{phaseEq}) for $K$ elements which is not possible
if $N<K$. In order to solve this problem, we need to make a case
differentiation and use the element stoichiometry of the consensates
to formulate our inner boundary condition as
\begin{center}
\begin{tabular}{lcl}
  1) element not affected by condensation &:& zero-flux boundary
                                              condition $j_k=0$,\\
  2) limiting element &:& fixed concentration $\ek$ from
            Eq.~(\ref{phaseEq}),\\
  3) non-limiting element &:& derived flux $j_k\neq0$  \ .
\end{tabular}
\end{center}




\section{Updating the crust thickness and composition}

\section{Element Conservation}



\bibliographystyle{chicagon}
\bibliography{references}

\end{document}
