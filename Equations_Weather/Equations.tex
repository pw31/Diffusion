\documentclass[11pt]{article}
\usepackage{graphicx,natbib,amsmath}

\setlength{\textheight}{24.5cm}
\setlength{\textwidth}{16.5cm}
\setlength{\topmargin}{-14mm}
\setlength{\evensidemargin}{-2mm}
\setlength{\oddsidemargin}{-2mm}
\setlength{\headsep}{4mm}

\def\nH{n_{\langle\rm H\rangle}}
\def\Vl{V_{\ell}}
\def\Nl{N_{\ell}}
\def\rhod{\rho_{\rm d}}
\def\ek{\epsilon_k}
\def\pdiff#1#2{\frac{\partial #1}{\partial #2}}
\def\rhogr{\rho_{\rm gr}}
\def\Dd{D_{\!d}}

\def\apj{ApJ}
\def\pasp{PASP}
\def\pasj{PASJ}
\def\araa{ARA\&A}
\def\aap{A\&A}
\def\aaps{A\&AS}
\def\apjl{ApJL}
\def\apjs{ApJS}
\def\mnras{MNRAS}
\def\aj{AJ}
\def\nat{Nature}
\def\icarus{Icarus}
\def\pasa{PASA}

\def\la{\mathrel{\mathchoice {\vcenter{\offinterlineskip\halign{\hfil
$\displaystyle##$\hfil\cr<\cr\sim\cr}}}
{\vcenter{\offinterlineskip\halign{\hfil$\textstyle##$\hfil\cr
<\cr\sim\cr}}}
{\vcenter{\offinterlineskip\halign{\hfil$\scriptstyle##$\hfil\cr
<\cr\sim\cr}}}
{\vcenter{\offinterlineskip\halign{\hfil$\scriptscriptstyle##$\hfil\cr
<\cr\sim\cr}}}}}
\def\ga{\mathrel{\mathchoice {\vcenter{\offinterlineskip\halign{\hfil
$\displaystyle##$\hfil\cr>\cr\sim\cr}}}
{\vcenter{\offinterlineskip\halign{\hfil$\textstyle##$\hfil\cr
>\cr\sim\cr}}}
{\vcenter{\offinterlineskip\halign{\hfil$\scriptstyle##$\hfil\cr
>\cr\sim\cr}}}
{\vcenter{\offinterlineskip\halign{\hfil$\scriptscriptstyle##$\hfil\cr
>\cr\sim\cr}}}}}


\begin{document}

\paragraph{Definitions}{\ }\\

\begin{tabular}{l|l|c}
\hline
symbol  & description & unit\\
\hline
$z$                       & vertical coordinate    & cm\\
$\nH$                     & hydrogen nuclei density & cm$^{-3}$\\
$\rho = \mu_{\rm H}\,\nH$  & gas mass density       & g\,cm$^{-3}$\\
$T$                       & gas temperature        & K\\
%$p = \frac{\rho}{\mu}kT$  & gas pressure           & dyn\,cm$^{-2}$\\
%$\mu$                     & mean gas particle mass & g\\
$a$                       & dust particle radius   & cm\\
$V=\frac{4\pi}{3}a^3$     & volume of a dust particle & cm$^3$\\
$\Vl$                     & miminum volume of a dust particle & cm$^3$\\
$f(V)$                    & size distribution function & $\rm cm^{-6}$\\
$\rho L_j=\int_{V_\ell}^\infty f(V)V^{j/3}\,dV$  
                          & dust moments           & cm$^{j-3}$\\
$L_j$                     & dust moments           & cm$^j$\,g$^{-1}$\\
$J_\star$                  & nucleation rate        & $\rm cm^{-3}s^{-1}$\\
$D$                       & diffusion coefficient  & cm$^2$\,s$^{-1}$\\
\hline  
\end{tabular}

\bigskip
\section{General Equations}

The dust moment equations are given by \citep[see derivation
  in][section~3]{Woitke2003}
\begin{equation}
  \pdiff{(\rho L_j)}{t} + \nabla(\vec{v}\,\rho L_j) 
    ~=~ \underbrace{\Vl^{j/3} \sum\limits_s J_\star^s}_{\rm nucleation}
    ~+\underbrace{\frac{j}{3}\,\chi\,\rho L_{j-1}}_{\rm growth}
    ~+\underbrace{\nabla\left(\Dd\,\rho\nabla L_j\right)}_{\rm diffusion}
    ~+\,\underbrace{\nabla\!\left(\xi\frac{\rhod}{c_T}L_{j+1}
                  \,\widehat{r}\right)}_{\rm drift}
\end{equation}
where the first term on the r.h.s. accounts for the increase of the
dust moments due to nucleation, the second term accounts for the
growth and evaporation of the dust particles ($\chi$ can be negative),
the third term is for diffusive mixing, and the forth term accounts
for the downward equilibrium drift of the grains in a gravitational
field, $\xi=\sqrt[3]{\frac{3}{4\pi}}\frac{\sqrt{\pi}}{2}\,g$, for the
case of small Knudsen numbers, see \citet[][section
  5.1]{Woitke2003}. $\widehat{r}$ is the unit vector away from the
center of gravity.  The idea of diffusive mixing due to convection and
overshoot is introduced in \citet[][Sect.~2.1 and, in
particular, Sect.~4.2]{Woitke2004}.

All dust grains are assumed to be perfect spheres with
well-mixed material composition, independent of size.
We can then introduce $\rho L_3^s=\int_{V_\ell}^\infty f(V)V^s\,dV$
\citep{Helling2008} and write
\begin{equation}
  V = \sum\limits_s V^s \quad\mbox{,}\quad
  L_3 = \sum\limits_s L_3^s \quad\mbox{,}\quad
  b^s = \frac{L_3^s}{L_3}  \ .
\end{equation}
In the following, $s=1\,...\,S$ is an index running over the
different solid species, $r=1\,...\,R$ is an index running over the 
surface reactions and $j=\{0,1,2,3,4\}$ is the index of the dust moments.
We introduce the following abbreviation for the surface reaction rates
[number of net reactions per surface area and per time]
\begin{equation}
  c_r^s ~=~ \sqrt[3]{36\pi}\;
          \frac{\nu_r^s\,n_r^{\rm key}v_r^{\rm rel}\,\alpha_r}{\nu_r^{\rm key}}
          \left(1-\frac{b^s}{S_r}\right) \ ,
\end{equation}
where $n_r^{\rm key}$ is the particle density [cm$^{-3}$] of the key
species of surface reaction $r$, $\nu_r^{\rm key}$ its stoichiometric
factor in that reaction, $v_r^{\rm rel}=\sqrt{kT/(2\pi\,m^{\rm key})}$
its thermal relative velocity, $m^{\rm key}$ its mass, $\alpha_r$ its
sticking probability, and $S_r$ is the reaction supersaturation
ratio. For example, in this reaction
\begin{equation}
 \rm 2\,CaH ~+~ 2\,Ti ~+~ 6\,H_2O ~\longrightarrow~ 2\,CaTiO_3[s] ~+~ 7\,H_2
\end{equation}
the key species is either key$=$CaH or key$=$Ti (depending on which
species is less abundant), $\nu_r^{\rm key}\!=\!2$ in both cases,
$s\!=\rm\!CaTiO_3$[s] is the solid species, and $\nu_r^s\!=\!2$ units
of CaTiO$_3$[s] are produced by one reaction. Using this definition,
we can express the net growth velocity of all grains $\chi$ [cm/s] as
\begin{equation}
  \chi = \sum\limits_s \sum\limits_r c_r^s V_0^s \ ,
\end{equation}
where $V_0^s$ is the volume of one solid unit of material kind $s$.
We can now separate the third moment equation and can express the 
element conservation equations as follows
\begin{eqnarray}
  \pdiff{(\rho L_3^s)}{t} + \nabla(\vec{v}\,\rho L_3^s) 
  &\!\!=\!\!& \Vl J_\star^s 
    ~+ \rho L_2 \sum\limits_r V_0^s c_r^s  
    ~+ \nabla\left(\Dd\,\rho\nabla L_3^s\right)
    ~+ \nabla\!\left(\xi\frac{\rhod}{c_T}\,b^s L_4\,\widehat{r}\right)\\
  \pdiff{(\nH\ek)}{t} + \nabla(\vec{v}\,\nH\ek)
  &\!\!=\!\!& -\sum\limits_s \nu_k^s \Nl J_\star^s 
    ~- \rho L_2 \sum\limits_s\sum\limits_r \nu_k^s c_r^s 
    ~+ \nabla\left(D\,\nH\nabla \ek\right)  \ ,
\end{eqnarray}
where $\nu_k^s$ is the stoichiometric factor of element $k$ in solid
$s$, for example $\nu_{\rm Ti}^{\rm TiO2[s]}=1$ and $\nu_{\rm O}^{\rm
  TiO2[s]}=2$. 

\section{Static planeparallel atmosphere}

Next we assume that the gas in the atmosphere is static $\vec{v}=0$ 
and that all processes are stationary $\pdiff{\,\cdot}{t}=0$. Also
assuming plane-parallel geometry $\nabla\to\frac{d}{dz}$, 
we find
\begin{eqnarray}
 0 &=& \Vl^{j/3} \sum\limits_s J_\star^s
    ~+ \frac{j}{3}\,\chi\,\rho L_{j-1}
    ~+ \frac{d}{dz}\!\left(\Dd\,\rho\frac{d L_j}{dz}\right)
    ~+ \frac{d}{dz}\!\left(\xi\frac{\rhod}{c_T}L_{j+1}\right) \\
 0 &=& \Vl J_\star^s 
    ~+ \rho L_2 \sum\limits_r V_0^s c_r^s  
    ~+ \frac{d}{dz}\!\left(\Dd\,\rho \frac{dL_3^s}{dz}\right)
    ~+ \frac{d}{dz}\!\left(\xi\frac{\rhod}{c_T}\,b^s L_4\right)\\
 0 &=& -\sum\limits_s \nu_k^s \Nl J_\star^s 
    ~- \rho L_2 \sum\limits_s\sum\limits_r \nu_k^s c_r^s 
    ~+ \frac{d}{dz}\!\left(D\,\nH\frac{d\ek}{dz}\right)  \ .
\end{eqnarray}

\subsection{Dust mixing -- yes or no?}

The assumed mixing by turbulence is described by a diffusion
approximation where the effective diffusion coefficients is much
larger than the gas-kinetic one 
\begin{equation}
  D_{\rm micro} ~=~ \frac{1}{3}\frac{v_{\rm th}}{\sigma\,n}
\end{equation}
where $1/(\sigma\,n)$ is the mean free path, $n$ would be the total
gas particle density and $\sigma\approx 3\times 10^{-15}\rm\,cm^2$
would be a typical cross-section for gas-gas collisions
\citep{Woitke2003}.  Instead, the turbulent
diffusion coefficient is roughly given by
\begin{equation}
  D ~\approx~ \langle v_z\rangle\,H_p ~\gg~  D_{\rm micro} \ ,
\end{equation} 
where $\langle v_z\rangle$ is the root-mean-square average of vertical
velocities in the atmosphere, at height $z$. In the convective layer,
$\langle v_z\rangle\approx v_{\rm conv}$ is the convective velocity
which is a part of the stellar atmosphere model and results from the
application of mixing length theory.  Above the convective layer,
where the Schwarzschild criterion for convection is false, $\langle
v_z\rangle$ will decrease rapidly with increasing $z$, but will not be
entirely zero due to convective overshoot. We apply a powerlaw
in $\log p$ to approximate this behaviour
\begin{equation}
  \log \langle v_z\rangle = \log v_{\rm conv} 
                          - \beta\cdot\max\{0,\log p_{\rm conv}-\log p(z)\}
\end{equation}
with free parameter $\beta\approx 1.5\,...\,2.2$ \citep{Ludwig2002,Lee2015}.
Due to their inertia, dust particles are less effected by turbulence,
namely only via the slower and bigger turbulent modes (larger eddies). The dust
diffusion coefficient is hence expected to be bracketed by the gas
diffusion coefficient as
\begin{equation}
  0 ~<~ \Dd ~\leq~ D \ .
\end{equation}
We cannot fully account for these effects, because we need an average
dust diffusion coefficient which we can pull out of the dust size
integrals. However, we can consider the following two limiting cases
\begin{equation}
  \begin{array}{cc}
  \mbox{{\bf case 1:}\ \ small grains\ \ } & \Dd=D\\
  \mbox{{\bf case 2:}\ \ large grains\ \ } & \Dd=0
  \end{array}
\end{equation}
For case 2, we are left with
\begin{eqnarray}
 0 &=& \Vl^{j/3} \sum\limits_s J_\star^s
    ~+ \frac{j}{3}\,\chi\,\rho L_{j-1}
    ~+ \frac{d}{dz}\!\left(\xi\frac{\rhod}{c_T}L_{j+1}\right) \\
 0 &=& \Vl J_\star^s 
    ~+ \rho L_2 \sum\limits_r V_0^s c_r^s  
    ~+ \frac{d}{dz}\!\left(\xi\frac{\rhod}{c_T}\,b^s L_4\right)
   \label{eq1}\\
 0 &=& -\sum\limits_s \nu_k^s \Nl J_\star^s 
    ~- \rho L_2 \sum\limits_s\sum\limits_r \nu_k^s c_r^s 
    ~+ \frac{d}{dz}\!\left(D\,\nH\frac{d\ek}{dz}\right)  \ .
   \label{eq2}
\end{eqnarray}
\subsection{The total flux of elements}

An important realisation is that in the static case, the total
vertical flux of elements (due to vertical dust settling {\it and\,} due to
turbulent mixing) must be zero everywhere in the atmosphere and for
each element. We can derive this conclusion formally by summing up
(Eq.~\ref{eq2}) and $\sum_s$
(Eq.~\ref{eq1})\,$\cdot\,\nu_k^s/V_0^s$. The nucleation and growth
terms cancel out exactly in this case, and we find
\begin{equation}
  \frac{d}{dz}\!\left(D\,\nH\frac{d\ek}{dz}\right) 
  ~+~ \frac{d}{dz}\!\left(\xi
      \frac{\rhod}{c_T}L_4\sum\limits_s\frac{\nu_k^s\,b^s}{V_0^s}\right)
  ~=~0  \ ,
\end{equation}
thus,
\begin{equation}
  D\,\nH\frac{d\ek}{dz} 
  ~+~ \xi\frac{\rhod}{c_T}L_4\sum\limits_s\frac{\nu_k^s\,b^s}{V_0^s}
  ~=~ \mbox{const}_k
\end{equation}
This equation stills allow for constant (time-independent and
height-independent) fluxes of elements through the atmosphere. Since
this case would require very strange boundary conditions (where do
these elements come from / where do they go?) we exclude that
possibility here and demand const$_k=0$ instead, finding
\begin{equation}
  \underbrace{D\,\nH\frac{d\ek}{dz}}_{\displaystyle-j_k^{\rm mix}} ~+~
  \underbrace{\xi\frac{\rhod}{c_T}L_4\sum\limits_s\frac{\nu_k^s\,b^s}{V_0^s}
             }_{\displaystyle-j_k^{\rm drift}}
  ~=~ 0
  \label{eq3}
\end{equation}
$j_k^{\rm drift}<0$ is the element flux [1/cm$^2$/s] due to the
downward gravitational settling of all dust particles.
Equation~(\ref{eq3}) means that this flux must be balanced by an
upward directed diffusive mixing flux $j_k^{\rm mix}>0$ due to turbulence, at
each height $z$ in the atmosphere and for all elements $k$. We conclude
\begin{equation}
  \frac{d\ek}{dz} ~=~
  -\frac{\xi\rhod L_4}{c_T D\,\nH} \sum\limits_s\frac{\nu_k^s\,b^s}{V_0^s}
  \quad\quad\leq~0
  \label{eq4}
\end{equation}
Whenever dust is present $(L_4>0)$ and gravity is active $(\xi>0)$,
the gas element gradients must be negative, i.e.\ the abundance of all
elements $k$ involved in dust formation {\sl must monotonically
  decrease} toward the top of the atmosphere.

\clearpage
\section{Solution method}

Using Eq.~(\ref{eq4}), we can write the system of coupled
differential equations for dust and gas as
\begin{equation}
\boxed{\begin{array}{rll}
  \displaystyle
  -\frac{d}{dz}\!\left(\xi\frac{\rhod}{c_T}L_{j+1}\right) 
  &=~\displaystyle
       \Vl^{j/3} \sum\limits_s J_\star^s
    ~+ \frac{j}{3}\,\chi\,\rho L_{j-1}
  &\quad(j=0,1,2)\\[5mm]
  \displaystyle
  -\frac{d}{dz}\!\left(\xi\frac{\rhod}{c_T}\,b^s L_4\right) 
  &=~\displaystyle
     \Vl J_\star^s 
    ~+ \rho L_2 \sum\limits_r V_0^s c_r^s
  &\quad(s=1,...\,,S)\\[5mm]
  \displaystyle
  -\frac{d\ek}{dz} 
  &=~\displaystyle
     \xi\frac{\rhod}{c_T}\frac{L_4}{D\,\nH} 
     \sum\limits_s\frac{\nu_k^s\,b^s}{V_0^s}
  &\quad(k=1,...\,,K) 
\end{array}}
\label{eqsystem}
\end{equation}

\subsection{Closure condition}

The moment equation system is completed by a closure condition of the form
\begin{equation}
  L_0 = {\cal F}(L_1,L_2,L_3,L_4)     \ ,
\end{equation}
see \citet[][section 2.4.1]{Woitke2004}.

\subsection{ODE system}

Our equation system (\ref{eqsystem}) has the standard form of a system
of Ordinary Differential Equations (ODE) as 
\begin{equation}
  \frac{d\vec{y}}{dx} ~=~ \vec{F}(x,\vec{y}) \ .
\end{equation}
We will denote $\vec{y}$ as the {\sl solution vector} and $\vec{F}$ as the
{\sl right-hand-side vector} in the following. The solution vector has
dimension $(3+S+K)$ and is given by
\begin{equation}
  \vec{y} = \{\xi\frac{\rhod}{c_T}L_1\ ,\ 
              \xi\frac{\rhod}{c_T}L_2\ ,\ 
              \xi\frac{\rhod}{c_T}L_3\ ,\ 
              \xi\frac{\rhod}{c_T}b^1 L_4\ ,\ ...\ , \ 
              \xi\frac{\rhod}{c_T}b^S L_4\ ,\ 
              \epsilon_1\ ,\ ...\ , \  
              \epsilon_K
            \} 
\end{equation}

\subsection{Boundary Conditions}

\begin{itemize}
\item The element abundances are given deep inside the
      star, well below the cloud layers, say at $z=0$.
\item All dust quantities can be assumed to be zero both
      at the top and the bottom of the atmosphere
\item I seems to me that the equations can only be integrated
      from top to bottom, following the natural direction of the
      settling dust particles
\item I tried to solve the equations with a ODE-solver with a 
      shooting method, but it looks to be very unstable and not
      working so far.
\end{itemize}
Huston, we have a problem.

\bibliographystyle{chicagon}
\bibliography{references}

\end{document}
